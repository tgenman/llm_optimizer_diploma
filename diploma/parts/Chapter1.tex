\chapter{Problem Statement}
\label{sec:Chapter1} \index{Chapter1}

In the paradigm Idea→ Formula→ Code state the problem to find an optimal solution
\begin{itemize}
    \item Discuss the problem statement with your adviser.
    \item See the examples below and in past projects.
    \item Discuss terminology and notation. See [pdf] and [tex] with notations and a useful style file.
    \item At the beginning of the Problem statement, write a general problem description.
    \item Describe the elements of your problem statement:
    \begin{itemize}
        \item the sample set,
        \item its origin, or its algebraic structure,
        \item statistical hypotheses of data generation,
        \item conditions of measurements,
        \item restrictions of the sample set and its values,
        \item your model in the class of models,
        \item restrictions on the class of models,
        \item the error function (and its inference) or a loss function, a quality criterion,
        \item cross-validation procedure,
        \item restrictions to the solutions,
        \item external (industrial) quality criteria,
        \item the optimization statement as argmin.
    \end{itemize}
    \item Define the main termini: what is called the model, the solution, and the algorithm.
\end{itemize}

\hrulefill

Tips for problem statement
Introduce the proper terminology. Note that:
\begin{itemize}
    \item The model is a parametric family of functions that map design space to target space.
    \item The criterion (error function, metric) is a function to optimize and get an optimal solution (model parameters, a function).
    \item The algorithm transforms solution space, usually iteratively.
    \item The method combines a model, a criterion, and an algorithm to produce a solution. Check it:
    \begin{itemize}
        \item the regression model,
        \item the sum of squared errors,
        \item the Newton-Raphson algorithm,
        \item the method of least squares.
    \end{itemize}
\end{itemize}

\hrulefill

In this thesis, we consider a standard \textbf{unconstrained minimization} problem in the context of linear regression. Let
\[
  \{(x^{(1)}, y^{(1)}), (x^{(2)}, y^{(2)}), \dots, (x^{(n)}, y^{(n)})\}
\]
be a set of $n$ training examples, where each $x^{(i)} \in \mathbb{R}^d$ is a feature vector, and each $y^{(i)} \in \mathbb{R}$ is the corresponding target. We assume there exists an unknown parameter vector $w^* \in \mathbb{R}^d$ such that 
\[
  y^{(i)} \;=\; \langle x^{(i)}, \, w^* \rangle 
  \quad \text{(possibly up to additive noise).}
\]

Our goal is to recover or approximate $w^*$ by minimizing the empirical risk:
\begin{equation}
  \label{eq:empirical-risk}
  R_{w^*}(w) \;=\; \frac{1}{2n} \Bigl\|\mathbf{X}^{\top} w \;-\; \mathbf{X}^{\top} w^*\Bigr\|^2,
\end{equation}
where $\mathbf{X}$ is the $d \times n$ matrix whose columns are the vectors $x^{(i)}$. 
In classical optimization terms, we aim to solve:
\[
  \min_{w \,\in\, \mathbb{R}^d}\; R_{w^*}(w).
\]
When $\mathbf{X}$ has full column rank (or $x^{(i)}$ are sampled from a non-degenerate distribution), 
the function $R_{w^*}(w)$ is convex and differentiable.

\subsection{Iterative First-Order Methods}

Many first-order optimization methods solve
\[
  \min_{w}\; f(w)
\]
by repeatedly updating $w$ using the gradient $\nabla f(w)$. A \emph{linear first-order method} (LFOM) 
can be written generally as
\begin{equation}
  \label{eq:LFOM}
  w_{k+1} \;=\; w_0 \;+\; \sum_{i=0}^{k}\Gamma_k^i\,\nabla f\bigl(w_i\bigr),
\end{equation}
where each $\Gamma_k^i$ is a (diagonal) matrix that scales the $i$-th gradient. This broad family includes, for example, gradient descent, momentum methods, and conjugate gradient descent for quadratic objectives.

\subsection{Memory-Augmented Transformers (Memformers)}

Recent studies show that Transformers \textbf{ memory-augmented} (also known as \emph{Memformers}) can \emph{implement} such iterative updates \emph{in-context} within their forward pass. Specifically, a Memformer can store and reuse intermediate computations (akin to saving previous gradients or partial solutions) via attention-based memory registers. Consequently, one can view the forward computation of a Memformer as performing gradient-based updates on \eqref{eq:empirical-risk}, effectively solving the linear regression problem. 

Formally, if we denote the iteration at step $k$ by $w_k$, Memformers learn to realize updates of the form:
\[
  w_{k+1} \;=\; \text{update}\bigl(w_k,\;\nabla R_{w^*}(w_k),\;\text{memory}\bigr),
\]
where the \emph{memory} captures relevant quantities like past gradients. This parallels the LFOM expression in \eqref{eq:LFOM} but is carried out through a sequence of transformer layers and attention blocks.

\subsection{Unconstrained Quadratic Example}

In linear regression,
\[   R_{w^*}(w) \;=\; \frac{1}{2n}   \Bigl\|\mathbf{X}^\top w \;-\; \mathbf{X}^\top w^*\Bigr\|^2, \],
which is a convex quadratic function. Popular methods like gradient descent, momentum, or conjugate gradient all iteratively minimize this function. 
Memformers replicate these methods by encoding partial iterates of the optimization trajectory into the attention layers.

Thus, the learning goal optimization $w$ that minimizes \eqref{eq:empirical-risk} can be viewed as
\[
  \min_{w \,\in\, \mathbb{R}^d} \quad 
  \frac{1}{2n}\|\mathbf{X}^\top w - \mathbf{X}^\top w^*\|^2,
\]
and is solved \emph{in-context} by the Memformer's learned parameters and attention-based memory mechanism.


\subsection{Lemma 1 \cite{ahn2024transformers}.}
We consider an $L$-layer linear Transformer with parameter matrices 
\[
P_\ell = 
\begin{bmatrix}
\mathbf{B}_\ell = 0_{d \times d} & 0 \\
0 & 1
\end{bmatrix},
\quad
Q_\ell = 
-\begin{bmatrix}
\mathbf{A}_\ell & 0 \\
0 & 0
\end{bmatrix}, 
\quad \mathbf{A}_\ell, \mathbf{B}_\ell \in \mathbb{R}^{d \times d}.
\]
Let $y_\ell^{(n+1)}$ denote the $(d+1,\,n+1)$-entry of the $\ell$-th layer output matrix~$\mathbf{Z}_\ell$, i.e., 
\(
   y_\ell^{(n+1)} = [\mathbf{Z}_\ell]_{(d+1),\,(n+1)}
\)
for $\ell = 1, \dots, L$. Then
\[
   y_\ell^{(n+1)} \;=\; -\,\bigl\langle \mathbf{x}^{(n+1)}, \,\mathbf{w}_\ell^{\mathrm{gd}}\bigr\rangle,
\]
where the sequence $\{\mathbf{w}_\ell^{\mathrm{gd}}\}$ is initialized by $\mathbf{w}_{0}^{\mathrm{gd}} = 0$ and, for $\ell = 1, \dots, L - 1,$ follows
\[
   \mathbf{w}_{\ell+1}^{\mathrm{gd}}
   \;=\;
   \mathbf{w}_{\ell}^{\mathrm{gd}}
   \;-\;
   \mathbf{A}_\ell 
   \,\nabla R_{\mathbf{w}^*}\bigl(\mathbf{w}_{\ell}^{\mathrm{gd}}\bigr),
\]
with the empirical least-squares loss 
\[
   R_{\mathbf{w}^*}(w) 
   \;:=\; 
   \frac{1}{2n}\,\bigl\|\mathbf{X}^\top w - \mathbf{X}^\top w^*\bigr\|^2
   \;=\;
   \frac{1}{2n}\,
   (w - w^*)^\top \,\mathbf{X}\,\mathbf{X}^\top \,(w - w^*).
\]
